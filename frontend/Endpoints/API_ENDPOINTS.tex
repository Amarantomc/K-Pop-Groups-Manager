\documentclass[11pt]{article}
\usepackage[utf8]{inputenc}
\usepackage[spanish]{babel}
\usepackage{geometry}
\usepackage{booktabs}
\usepackage{longtable}
\usepackage{array}
\usepackage{xcolor}
\usepackage{hyperref}

\geometry{a4paper, landscape, margin=1.5cm}

\title{Documentación de Endpoints de la API}
\author{IS-K Pop Backend}
\date{30 de noviembre de 2025}

\begin{document}

\maketitle

\section*{Información General}

\textbf{Base URL:} \texttt{http://localhost:\{PORT\}/api}

\section{Autenticación (\texttt{/auth})}

\begin{longtable}{lllll}
\toprule
\textbf{Método} & \textbf{URL} & \textbf{Autenticación} & \textbf{Rol Requerido} & \textbf{Descripción} \\
\midrule
\endhead
POST & \texttt{/api/auth/login} & No & - & Iniciar sesión \\
\bottomrule
\end{longtable}

\subsection*{POST /api/auth/login - Estructura}

\textbf{Request Body:}
\begin{verbatim}
{
  "email": "usuario@gmail.com",
  "password": "password123"
}
\end{verbatim}

\textbf{Response Exitosa (200):}
\begin{verbatim}
{
  "success": true,
  "data": {
    "user": {
      "id": 1,
      "email": "usuario@gmail.com",
      "name": "Nombre Usuario",
      "role": "admin",
      "profileData": {
        "agenciaId": 1  // Para manager/director
        // o "aprendizId": 1  // Para apprentice
        // o "artistaIdAp": 1, "artistaIdGr": 1  // Para artist
      }
    },
    "token": "eyJhbGciOiJIUzI1NiIsInR5cCI6IkpXVCJ9...",
    "expiresIn": 3600
  }
}
\end{verbatim}

\textbf{Response Error (401):}
\begin{verbatim}
{
  "success": false,
  "error": "Invalid credentials"
}
\end{verbatim}

\section{Usuarios (\texttt{/user})}

\begin{longtable}{lllll}
\toprule
\textbf{Método} & \textbf{URL} & \textbf{Autenticación} & \textbf{Rol Requerido} & \textbf{Descripción} \\
\midrule
\endhead
POST & \texttt{/api/user} & Sí & Admin & Crear usuario \\
GET & \texttt{/api/user} & Sí & - & Listar usuarios \\
GET & \texttt{/api/user/:id} & Sí & - & Obtener usuario por ID \\
PUT & \texttt{/api/user/:id} & Sí & Admin & Actualizar usuario \\
DELETE & \texttt{/api/user/:id} & Sí & Admin & Eliminar usuario \\
\bottomrule
\end{longtable}

\subsection*{POST /api/user/ - Crear Usuario}

\textbf{Headers:} \texttt{Authorization: Bearer \{token\}}

\textbf{Request Body - Admin:}
\begin{verbatim}
{
  "email": "admin@test.com",
  "name": "Admin Usuario",
  "password": "password123",
  "role": "admin"
}
\end{verbatim}

\textbf{Request Body - Manager/Director:}
\begin{verbatim}
{
  "email": "manager@test.com",
  "name": "Manager Usuario",
  "password": "password123",
  "role": "manager",
  "agencyId": 1
}
\end{verbatim}

\textbf{Request Body - Apprentice:}
\begin{verbatim}
{
  "email": "apprentice@test.com",
  "name": "Aprendiz Usuario",
  "password": "password123",
  "role": "apprentice",
  "IdAp": 1
}
\end{verbatim}

\textbf{Request Body - Artist:}
\begin{verbatim}
{
  "email": "artist@test.com",
  "name": "Artista Usuario",
  "password": "password123",
  "role": "artist",
  "IdAp": 1,
  "IdGr": 1
}
\end{verbatim}

\textbf{Response Exitosa (201):}
\begin{verbatim}
{
  "success": true,
  "data": {
    "id": 5,
    "email": "usuario@test.com",
    "name": "Usuario",
    "role": "manager",
    "profileData": { "agenciaId": 1 }
  }
}
\end{verbatim}

\textbf{Response Error (400):}
\begin{verbatim}
{
  "success": false,
  "error": "Missing required fields"
}
\end{verbatim}

\subsection*{GET /api/user/ - Listar Usuarios}

\textbf{Response Exitosa (200):}
\begin{verbatim}
{
  "success": true,
  "data": [
    {
      "id": 1,
      "email": "admin@gmail.com",
      "name": "Admin Usuario",
      "role": "admin",
      "profileData": {}
    }
  ]
}
\end{verbatim}

\subsection*{GET /api/user/:id - Obtener Usuario}

\textbf{Response Exitosa (200):}
\begin{verbatim}
{
  "success": true,
  "data": {
    "id": 1,
    "email": "usuario@test.com",
    "name": "Usuario",
    "role": "manager",
    "profileData": { "agenciaId": 1 }
  }
}
\end{verbatim}

\subsection*{PUT /api/user/:id - Actualizar Usuario}

\textbf{Request Body:} Campos opcionales a actualizar

\textbf{Response Exitosa (200):}
\begin{verbatim}
{
  "success": true,
  "data": { /* Usuario actualizado */ }
}
\end{verbatim}

\subsection*{DELETE /api/user/:id - Eliminar Usuario}

\textbf{Response Exitosa (200):}
\begin{verbatim}
{
  "success": true,
  "message": "User deleted successfully"
}
\end{verbatim}

\section{Agencias (\texttt{/agency})}

\begin{longtable}{lllll}
\toprule
\textbf{Método} & \textbf{URL} & \textbf{Autenticación} & \textbf{Rol Requerido} & \textbf{Descripción} \\
\midrule
\endhead
POST & \texttt{/api/agency} & Sí & Staff & Crear agencia \\
GET & \texttt{/api/agency} & Sí & - & Listar todas las agencias \\
GET & \texttt{/api/agency/:id} & Sí & - & Obtener agencia por ID \\
PUT & \texttt{/api/agency/:id} & Sí & Staff & Actualizar agencia \\
DELETE & \texttt{/api/agency/:id} & Sí & Staff & Eliminar agencia \\
GET & \texttt{/api/agency/search/agency\_name} & Sí & - & Buscar agencias por nombre \\
GET & \texttt{/api/agency/search/agency\_address} & Sí & - & Buscar agencias por dirección \\
GET & \texttt{/api/agency/search/agency\_foundation} & Sí & - & Buscar por fecha fundación \\
\bottomrule
\end{longtable}

\subsection*{POST /api/agency/ - Crear Agencia}

\textbf{Request Body:}
\begin{verbatim}
{
  "name": "SM Entertainment",
  "address": "Seúl, Corea del Sur",
  "foundation": "1995-02-14"
}
\end{verbatim}

\textbf{Response Exitosa (201):}
\begin{verbatim}
{
  "success": true,
  "data": {
    "id": 1,
    "nombre": "SM Entertainment",
    "ubicacion": "Seúl, Corea del Sur",
    "fechaFundacion": "1995-02-14T00:00:00.000Z"
  }
}
\end{verbatim}

\subsection*{GET /api/agency/ - Listar Agencias}

\textbf{Response Exitosa (200):}
\begin{verbatim}
{
  "success": true,
  "data": [
    {
      "id": 1,
      "nombre": "SM Entertainment",
      "ubicacion": "Seúl",
      "fechaFundacion": "1995-02-14T00:00:00.000Z"
    }
  ]
}
\end{verbatim}

\subsection*{GET /api/agency/:id - Obtener Agencia}

\textbf{Response Exitosa (200):} Mismo formato que GET individual

\subsection*{GET /api/agency/search/agency\_name?name=X}

\textbf{Query Params:} \texttt{name}

\textbf{Response:} Array de agencias que coincidan

\section{Aprendices (\texttt{/apprentice})}

\begin{longtable}{lllll}
\toprule
\textbf{Método} & \textbf{URL} & \textbf{Autenticación} & \textbf{Rol Requerido} & \textbf{Descripción} \\
\midrule
\endhead
POST & \texttt{/api/apprentice/:id} & Sí & Staff & Crear aprendiz \\
GET & \texttt{/api/apprentice} & Sí & - & Listar todos los aprendices \\
GET & \texttt{/api/apprentice/:id} & Sí & - & Obtener aprendiz por ID \\
GET & \texttt{/api/apprentice/:name} & Sí & - & Obtener aprendiz por nombre \\
GET & \texttt{/api/apprentice/agency/:id} & Sí & - & Listar aprendices por agencia \\
PUT & \texttt{/api/apprentice/:id} & Sí & Staff & Actualizar aprendiz \\
DELETE & \texttt{/api/apprentice/:id} & Sí & Staff & Eliminar aprendiz \\
\bottomrule
\end{longtable}

\subsection*{POST /api/apprentice/:id - Crear Aprendiz}

\textbf{URL Param:} \texttt{id} - ID de la agencia

\textbf{Request Body:}
\begin{verbatim}
{
  "name": "Kim Minju",
  "dateOfBirth": "2001-02-05",
  "age": 22,
  "trainingLv": 3,
  "status": "En entrenamiento"
}
\end{verbatim}

\textbf{Validaciones:}
\begin{itemize}
    \item \texttt{age} $\geq$ 15
    \item \texttt{trainingLv} $\geq$ 0
\end{itemize}

\textbf{Response Exitosa (201):}
\begin{verbatim}
{
  "success": true,
  "data": {
    "id": 1,
    "nombreCompleto": "Kim Minju",
    "fechaNacimiento": "2001-02-05T00:00:00.000Z",
    "edad": 22,
    "nivelEntrenamiento": 3,
    "estadoAprendiz": "En entrenamiento"
  }
}
\end{verbatim}

\subsection*{GET /api/apprentice/ - Listar Aprendices}

\textbf{Response Exitosa (200):}
\begin{verbatim}
{
  "success": true,
  "data": [ /* Array de aprendices */ ]
}
\end{verbatim}

\subsection*{GET /api/apprentice?name=X - Buscar por Nombre}

\textbf{Query Params:} \texttt{name}

\textbf{Response:} Aprendiz(es) que coincidan con el nombre

\section{Artistas (\texttt{/artist})}

\begin{longtable}{lllll}
\toprule
\textbf{Método} & \textbf{URL} & \textbf{Autenticación} & \textbf{Rol Requerido} & \textbf{Descripción} \\
\midrule
\endhead
POST & \texttt{/api/artist} & Sí & Staff & Crear artista \\
GET & \texttt{/api/artist} & Sí & Agency Access & Listar todos los artistas \\
GET & \texttt{/api/artist/:apprenticeId\&:groupId} & Sí & - & Obtener artista por IDs \\
GET & \texttt{/api/artist/:id} & Sí & - & Obtener artistas por agencia \\
PUT & \texttt{/api/artist/:apprenticeId\&:groupId} & Sí & Staff & Actualizar artista \\
DELETE & \texttt{/api/artist/:apprenticeId\&:groupId} & Sí & Staff & Eliminar artista \\
\bottomrule
\end{longtable}

\subsection*{POST /api/artist/ - Crear Artista}

\textbf{Request Body:}
\begin{verbatim}
{
  "apprenticeId": 1,
  "groupId": 1,
  "stageName": "Lisa",
  "debutDate": "2016-08-08",
  "status": "Activo"
}
\end{verbatim}

\textbf{Response Exitosa (201):}
\begin{verbatim}
{
  "success": true,
  "data": {
    "idAp": 1,
    "idGr": 1,
    "nombreArtistico": "Lisa",
    "fsechaDebut": "2016-08-08T00:00:00.000Z",
    "estadoArtista": "Activo"
  }
}
\end{verbatim}

\subsection*{GET /api/artist/ - Listar Artistas}

\textbf{Response Exitosa (200):}
\begin{verbatim}
{
  "success": true,
  "data": [ /* Array de artistas */ ]
}
\end{verbatim}

\subsection*{GET /api/artist/:apprenticeId\&:groupId - Obtener Artista}

\textbf{Nota:} Usa llave compuesta (apprenticeId y groupId)

\section{Conceptos (\texttt{/concept})}

\begin{longtable}{lllll}
\toprule
\textbf{Método} & \textbf{URL} & \textbf{Autenticación} & \textbf{Rol Requerido} & \textbf{Descripción} \\
\midrule
\endhead
POST & \texttt{/api/concept} & No & - & Crear concepto \\
GET & \texttt{/api/concept} & No & - & Listar todos los conceptos \\
GET & \texttt{/api/concept/:id} & No & - & Obtener concepto por ID \\
PUT & \texttt{/api/concept/:id} & No & - & Actualizar concepto \\
DELETE & \texttt{/api/concept/:id} & No & - & Eliminar concepto \\
\bottomrule
\end{longtable}

\subsection*{POST /api/concept/ - Crear Concepto}

\textbf{Request Body:}
\begin{verbatim}
{
  "description": "Concepto futurista y tecnológico"
}
\end{verbatim}

\textbf{Response Exitosa (201):}
\begin{verbatim}
{
  "success": true,
  "data": {
    "id": 1,
    "descripcion": "Concepto futurista y tecnológico"
  }
}
\end{verbatim}

\subsection*{GET /api/concept/ - Listar Conceptos}

\textbf{Response Exitosa (200):}
\begin{verbatim}
{
  "success": true,
  "data": [ /* Array de conceptos */ ]
}
\end{verbatim}

\section{Notas Importantes}

\subsection*{Estructura de Respuestas}

\textbf{Todas las respuestas exitosas:}
\begin{verbatim}
{
  "success": true,
  "data": { ... }  // o [ ... ] para arrays
}
\end{verbatim}

\textbf{Todas las respuestas con error:}
\begin{verbatim}
{
  "success": false,
  "error": "Mensaje de error descriptivo"
}
\end{verbatim}

\subsection*{Códigos HTTP}
\begin{itemize}
    \item \textbf{200 OK:} Operación exitosa (GET, PUT, DELETE)
    \item \textbf{201 Created:} Recurso creado (POST)
    \item \textbf{400 Bad Request:} Datos inválidos
    \item \textbf{401 Unauthorized:} Sin autenticación o credenciales inválidas
    \item \textbf{403 Forbidden:} Sin permisos
    \item \textbf{404 Not Found:} Recurso no encontrado
    \item \textbf{500 Internal Server Error:} Error del servidor
\end{itemize}

\subsection*{Autenticación}
Los endpoints que requieren autenticación necesitan un token JWT en el header:

\texttt{Authorization: Bearer \{token\}}

\subsection*{Roles Disponibles}
\begin{itemize}
    \item \textbf{Admin}: Acceso completo al sistema
    \item \textbf{Staff}: Director/Manager - puede gestionar agencias, aprendices y artistas
    \item \textbf{Agency Access}: Acceso relacionado con agencias específicas
\end{itemize}

\subsection*{Advertencia}
\textcolor{red}{\textbf{Conflicto detectado:}} En el archivo \texttt{index.ts} línea 23, las rutas de \texttt{/concept} están usando \texttt{artistRoutes.getRouter()} en lugar de \texttt{conceptRoutes.getRouter()}. Se recomienda corregir esto.

\subsection*{Configuración}
El puerto debe ser configurado en el archivo \texttt{secrets.ts}.

\end{document}
